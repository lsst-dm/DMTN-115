\addtocounter{table}{-1}
\begin{longtable}{|l|p{0.8\textwidth}|}\hline
\textbf{Acronym} & \textbf{Description}  \\\hline

API & Application Programming Interface \\\hline
AURA & Association of Universities for Research in Astronomy \\\hline
AWS & Amazon Web Services \\\hline
Butler & A middleware component for persisting and retrieving image datasets (raw or processed), calibration reference data, and catalogs. \\\hline
CAOM & Common Archive Observation Model \\\hline
CI & Continuous Integration \\\hline
CRUD & Create Retrieve Update and Destroy \\\hline
CSV & Comma Separated Values \\\hline
Center & An entity managed by AURA that is responsible for execution of a federally funded project \\\hline
DMTN & DM Technical Note \\\hline
Data Management & The LSST Subsystem responsible for the Data Management System (DMS), which will capture, store, catalog, and serve the LSST dataset to the scientific community and public. The DM team is responsible for the DMS architecture, applications, middleware, infrastructure, algorithms, and Observatory Network Design. DM is a distributed team working at LSST and partner institutions, with the DM Subsystem Manager located at LSST headquarters in Tucson. \\\hline
FITS & Flexible Image Transport System \\\hline
FSAAS & Filesystem as a Service \\\hline
FUSE & a user space filesystem framework \\\hline
HDF & Hierarchical Data Format \\\hline
HPC & High Performance Computing \\\hline
HTC & High Throughput Computing \\\hline
IRAF & Image Reduction and Analysis Facility \\\hline
JSON & JavaScript Object Notation \\\hline
LSST & Large Synoptic Survey Telescope \\\hline
Object & In LSST nomenclature this refers to an astronomical object, such as a star, galaxy, or other physical entity. E.g., comets, asteroids are also Objects but typically called a Moving Object or a Solar System Object (SSObject). One of the DRP data products is a table of Objects detected by LSST which can be static, or change brightness or position with time. \\\hline
POSIX & Portable Operating System Interface \\\hline
algorithm & A computational implementation of a calculation or some method of processing. \\\hline
metadata & General term for data about data, e.g., attributes of astronomical objects (e.g. images, sources, astroObjects, etc.) that are characteristics of the objects themselves, and facilitate the organization, preservation, and query of data sets. (E.g., a FITS header contains metadata). \\\hline
pipeline & A configured sequence of software tasks (Stages) to process data and generate data products. Example: Association Pipeline. \\\hline
provenance & Information about how LSST images, Sources, and Objects were created (e.g., versions of pipelines, algorithmic components, or templates) and how to recreate them. \\\hline
\end{longtable}
